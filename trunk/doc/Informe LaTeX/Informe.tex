\documentclass[a4paper,10pt]{article}

\usepackage{graphicx}
\usepackage[ansinew]{inputenc}
\usepackage[spanish]{babel}
\usepackage{pdfpages} % Para importar PDF 

\title{		\textbf{Trabajo Pr�ctico - Etapa 1 \\ \textit{Voto Electr�nico}} }

\author{	Mart�n Hern�n G�mez, \textit{Padr�n Nro. 85.780}                     \\
            \texttt{martinhgomez@yahoo.com.ar}                                              \\[2.5ex]
            Ignacio Marambio Cat�n, \textit{Padr�n Nro. 82.694}                     \\
            \texttt{ignacio.marambio@gmail.com}                                             \\[2.5ex]
            Mart�n Eduardo Quiroz, \textit{Padr�n Nro. 86.012}                     \\
            \texttt{martinedq@yahoo.com.ar}                                              \\[2.5ex]
            Daniel Shlufman, \textit{Padr�n Nro. 88.040}                     \\
            \texttt{incorporado@gmail.com}                                              \\[2.5ex]
            Lucas Damian Tarcetti, \textit{Padr�n Nro. 87.165}                     \\
            \texttt{lucas.tarcetti@yahoo.com.ar}                                              \\[2.5ex]
            \normalsize{2do. Cuatrimestre de 2011}                                      \\
            \normalsize{75.06 Organizaci�n de Datos  $-$ C�tedra Lic. Arturo Servetto}  \\
            \normalsize{Facultad de Ingenier�a, Universidad de Buenos Aires}            \\
       }
\date{26/10/11}

\begin{document}

\maketitle
\thispagestyle{empty}   % quita el n�mero en la primer p�gina
\newpage

\tableofcontents
\newpage

\section{Introducci�n}

El trabajo consiste en...
\\
y ademas...

\section{Objetivos}

El objetivo de este trabajo pr�ctico es...

\begin{itemize}
\item Sistema operativo Linux.
\item Lenguaje de programacion C++
\item Lenguaje de marcado \LaTeX.
\end{itemize}


\section{Desarrollo}

A continuaci�n...

\subsection{Instalaci�n}
Instalaci�n...

\subsection{Ejecuci�n}
Ejecuci�n:
\begin{quote}
voto -c /ArchivosAuxiliares/config.txt
\end{quote}

\textbf{Nota:} -c es configuraci�n



\section{Corridas de prueba}

A continuaci�n se detallan las pruebas realizadas sobre el funcionamiento del programa. Se emplearon las pruebas detalladas en el enunciado del informe, las cuales pasaron con �xito.

Corridas de prueba:

\begin{verbatim}

Ingrese su nombre de usuario: undomiel

Ahora ingrese su contrase�a: aragorn

usuario: <undomiel>
password: <aragorn>
INGRESO APROBADO
Bienvenido al sistama de gestion de elecciones
� Desea eliminar la base de datos y comenzar de 0? S/N
N

Opciones: 

1) Mantener Distritos
2) Mantener Votantes
3) Mantener Elecciones
4) Mantener Cargos
5) Mantener Listas
6) Mantener Candidatos
7) Informar Resultados
8) Habilitar Elecciones
9) Habilitar Votantes para elecci�n
10) salir
Opcion: 
.
.
.
\end{verbatim}


\newpage


\section{Conclusiones}
Conclusiones 
\newpage

\begin{thebibliography}{01}

\bibitem{NET01} (Wikipedia) \begin{verbatim}
http://en.wikipedia.org
\end{verbatim}

\end{thebibliography}
\newpage



\section{C�digo en C++}

A continuaci�n se adjunta el c�digo del programa en lenguaje C++

(nah, mentira)
% \includepdf[pages={1-3}]{./fuente.pdf}


\newpage



\section {Apendice - Enunciado TP Etapa 1}
A continuaci�n se agrega el enunciado de la Etapa 1, correspondiente a este Trabajo Pr�ctico.
% \includepdf[pages={1-4}]{./enunciado.pdf}


\end{document}

